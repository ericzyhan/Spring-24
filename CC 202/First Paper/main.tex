\documentclass[12pt]{article}

%
%Margin - 1 inch on all sides
%
\usepackage[letterpaper]{geometry}
% \usepackage{times}
\geometry{top=1.0in, bottom=1.0in, left=1.0in, right=1.0in}

%
%Doublespacing
%
\usepackage{setspace}
\doublespacing

%
%Rotating tables (e.g. sideways when too long)
%
\usepackage{rotating}

\usepackage[utf8]{inputenc}

%
%Fancy-header package to modify header/page numbering (insert last name)
%
\usepackage{fancyhdr}
\pagestyle{fancy}
\lhead{} 
\chead{} 
\rhead{\thepage} 
\lfoot{} 
\cfoot{} 
\rfoot{} 
\renewcommand{\headrulewidth}{0pt} 
\renewcommand{\footrulewidth}{0pt} 
%To make sure we actually have header 0.5in away from top edge
%12pt is one-sixth of an inch. Subtract this from 0.5in to get headsep value
\setlength\headsep{0.333in}

%
%Works cited environment
%(to start, use \begin{workscited...}, each entry preceded by \bibent)
% - from Ryan Alcock's MLA style file
%
\newcommand{\bibent}{\noindent \hangindent 40pt}
\newenvironment{workscited}{\newpage \begin{center} Works Cited \end{center}}{\newpage }


%
%Begin document
%
\begin{document}
\begin{flushleft}

%%%%First page name, class, etc
Eric Han\\
Professor Anita Patterson\\
CC 202\\
\today \\


%%%%Title
\begin{center}
On Freedom in Kant's \textit{Groundwork}
\end{center}


%%%%Changes paragraph indentation to 0.5in
\setlength{\parindent}{0.5in}
%%%%Begin body of paper here

Freedom in Kant's \textit{Groundwork} is extreme in its interpretation of human autonomy, opting to view freedom as an expression of rational principles rather than a will to be exercised in the pursuit of self-actualization. While Kant writes from a constructivist \textit{a priori} standpoint, the effectiveness and applicability of his model ethics are difficult to rigorously observe from a realist perspective. Additionally, in a truly enlightened society \\
Must write about \textit{freedom} first.\\
In \textit{Groundwork}, Kant plays the role of a moral extremist, placing duty and the moral worth of one's actions above all else. While this approach guarantees the homogeneity of society's ethics, it also places a disproportionate amount of emphasis on society's innate morality. Kant argues, under the basis of universal maxims, ``...all maxims are repudiated that are inconsistent with the will's own giving of universal law.'' (Kant \textit{Groundwork}, 81), saying that morally corrupt or valueless maxims will naturally raise a logical inconsistency with the (good) will's universal law. This statement is made under two central assumptions: that an incorruptible good will is eminent in the individual, and that they are rational. However, realistically, one cannot really attest to any universal rationality or a steadfast, unmoldable good will. Herein lies a central contradiction of Kant's: actions only have true moral value when they are borne from duty - that is, from the maxims one imposes on oneself - and freedom lies in our autonomy to act in accordance to duty. However, no society is truly homogenous, in that good will and rationality will not exist in every actively participating member of society. Therefore, the assumption of good will and rationality cannot be truly observed in any realist interpretation of Kant's \textit{Groundwork}, and the idea of a ``universal law'' falls apart. This contradiction is dangerous for a few reasons: first, in that the idea of moral worth as adherent to duty can be twisted in such a way that the consequences of duty are far more dire than the alternative and second, that freedom is bound to the same dangerous, rigid idea of duty that moral value is bound to. \\
In tackling the first consequence of this contradiction, it's important to review what Kant considers ``immoral''. The majority of his stance is based off of the faultiness of the hypothetical imperative, stating ``Practical good...is that which determines the will by means of representations of reason, hence not by subjective causes...'' (Kant \textit{Groundwork}, 67). That is, an action is only morally valuable if substantiated from reason alone. Initially, this doesn't seem to be any cause for concern, if only because Kant operates under the assumption of a perfectly rational being unruled by emotion, experience, or external phenomena. However, this definition of ``practical good'' can be jeapordized by a somewhat simple example. Suppose there exists a greedy king ruled perfectly by rationality. The act of protecting their people is not ruled by inclination (i.e, \textit{if} I want to continue collecting taxes, \textit{then} I must protect my people), but rather by the innate moral duty instated by being their ruler.
\newpage


%%%%Title
\begin{center}
Notes
\end{center}


\setlength{\parindent}{0.5in}

1. Danhof includes “Delaware, Maryland, all states north of the Potomac and Ohio rivers, Missouri, and states to its north” when referring to the northern states (11).


2. For the purposes of this paper,“science” is defined as it was in nineteenthcentury agriculture: conducting experiments and engaging in research.


3. Please note that any direct quotes from the nineteenth century texts are writtenin their original form, which may contain grammar mistakes according to twenty-first century grammar rules.

%%%%Works cited
\begin{workscited}

\bibent
Allen, R.L. \textit{The American Farm Book; or Compend of Ameri can Agriculture; Being a Practical Treatise on Soils, Manures, Draining, Irrigation, Grasses, Grain, Roots, Fruits, Cotton, Tobacco, Sugar Cane, Rice, and Every Staple Product of the United States with the Best Methods of Planting, Cultivating, and Prep aration for Market.} New York: Saxton, 1849. Print.

\bibent
Baker, Gladys L., Wayne D. Rasmussen, Vivian Wiser, and Jane M. Porter. \textit{Century of Service: The First 100 Years of the United States Department of Agriculture.}[Federal Government], 1996. Print.

\bibent
Danhof, Clarence H. \textit{Change in Agriculture: The Northern United States, 1820-1870.} Cambridge: Harvard UP, 1969. Print.


\end{workscited}

\end{flushleft}
\end{document}
\}