\documentclass[12pt]{article}

%
%Margin - 1 inch on all sides
%
\usepackage[letterpaper]{geometry}
% \usepackage{times}
\geometry{top=1.0in, bottom=1.0in, left=1.0in, right=1.0in}

%
%Doublespacing
%
\usepackage{setspace}
\doublespacing

%
%Rotating tables (e.g. sideways when too long)
%
\usepackage{rotating}

\usepackage[utf8]{inputenc}

%
%Fancy-header package to modify header/page numbering (insert last name)
%
\usepackage{fancyhdr}
\pagestyle{fancy}
\lhead{} 
\chead{} 
\rhead{\thepage} 
\lfoot{} 
\cfoot{} 
\rfoot{} 
\renewcommand{\headrulewidth}{0pt} 
\renewcommand{\footrulewidth}{0pt} 
%To make sure we actually have header 0.5in away from top edge
%12pt is one-sixth of an inch. Subtract this from 0.5in to get headsep value
\setlength\headsep{0.333in}

%
%Works cited environment
%(to start, use \begin{workscited...}, each entry preceded by \bibent)
% - from Ryan Alcock's MLA style file
%
\newcommand{\bibent}{\noindent \hangindent 40pt}
\newenvironment{workscited}{\newpage \begin{center} Works Cited \end{center}}{\newpage }


%
%Begin document
%
\begin{document}
\begin{flushleft}

%%%%First page name, class, etc
Eric Han\\
Professor Anita Patterson\\
CC 202\\
\today \\


%%%%Title
\begin{center}
On Freedom in Kant's \textit{Groundwork}
\end{center}


%%%%Changes paragraph indentation to 0.5in
\setlength{\parindent}{0.5in}
%%%%Begin body of paper here

Immanuel Kant's \textit{Groundwork of the Metaphysics of Morals} is a text which explores the innate human will and its ability to enact actions with good intentions. In this text, Kant rigorously explores the nature of humanity through definitions of will and moral value, and raises an uncompromising argument for what truly comprises moral value, freedom, and ethics. However, freedom in Kant's \textit{Groundwork} is extreme in its interpretation of human autonomy, opting to view freedom as an expression of rational principles rather than a will to be exercised in the pursuit of self-actualization. While Kant writes from a constructivist \textit{a priori} standpoint, the effectiveness and applicability of his model ethics are difficult to observe from a realist perspective.\\
In \textit{Groundwork}, Kant plays the role of a moral extremist, placing duty and the moral worth of one's actions above all else. While this approach guarantees the homogeneity of society's ethics, it also places a disproportionate amount of emphasis on society's innate morality. Kant argues, under the basis of universal maxims, ``...all maxims are repudiated that are inconsistent with the will's own giving of universal law.'' (Kant \textit{Groundwork}, 81), saying that morally corrupt or valueless maxims will naturally raise a logical inconsistency with the (good) will's universal law. This statement is made under two central assumptions: that an incorruptible good will is eminent in the individual, and that they are rational. However, realistically, one cannot really attest to any universal rationality or a steadfast, unmoldable good will. Herein lies a central contradiction of Kant's: actions only have true moral value when they are borne from duty - that is, from the maxims one imposes on oneself - and freedom lies in our autonomy to act in accordance to duty. However, no society is truly homogenous, in that good will and rationality will not exist in every actively participating member of society. Therefore, the assumption of good will and rationality cannot be truly observed in any realist interpretation of Kant's \textit{Groundwork}, and the idea of a ``universal law'' falls apart. This contradiction is dangerous for a few reasons: first, in that the idea of moral worth as adherent to duty can be twisted in such a way that the consequences of duty are far more dire than the alternative and second, that freedom is bound to the same dangerous, rigid idea of duty that moral value is bound to. \\
In tackling the first consequence of this contradiction, it's important to review what Kant considers ``immoral''. The majority of his stance is based off of the faultiness of the hypothetical imperative, stating ``Practical good...is that which determines the will by means of representations of reason, hence not by subjective causes...'' (Kant \textit{Groundwork}, 67). That is, an action is only morally valuable if substantiated from reason alone. Initially, this doesn't seem to be any cause for concern, if only because Kant operates under the assumption of a perfectly rational being unruled by emotion, experience, or external phenomena. However, this definition of ``practical good'' can be jeapordized by a classical example. Suppose you exist in an imperfect world, where while you operate under the pretense of your maxims, society as a whole is largely irrational. If you've been captured and your captors threaten your family, would it be correct to tell a small lie to save yourself and your family? By Kantian ethics, the categorical imperative would demand that you tell the truth, ultimately ending in your death and the death of your family. This is no complex moral situation, but abiding by rationality and duty results in the facilitation of objectively moral evil and the deaths of innocent people. \\
Alternatively, one can argue that the maxim of always being truthful is too rigid, and you may lie when the truth endangers innocents. However, this again is defeated by another example. Imagine the owner of a well. He directly distributes water to a village down the hill. However, he is one day approached by refugees seeking water. The well has enough water to sustain the village, and just enough for natural emergencies if they should strike the village. If he lies to the refugees and tells them that there is no water in the well, he is harming the innocent refugees. If he offers the refugees water, then he is harming the innocent people of the village should a disaster strike. Is there a clear moral maxim that triumphs here? There is no guilty party here, nor is there a clear outlook to the consequences of either choice. Kantian ethics don't apply to this complex situation, and modifying one's maxims over and over to adapt to different situations may eventually result in ethics based more off of personal moral values (what Kant may call inclination) rather than some universal duty. Kant's emphasis on duty and rational principles provide a framework that is too rigorous to be applied to real-world ethical dilemmas, leading to difficult situations where Kantian ethics are difficult, or even impossible, to apply while aligning to the broader, contemporary sense of morality.\\
In both former examples, adhering to Kantian ethics puts one in a difficult situation. Both the innocent victim of a kidnapping and the well owner for the village are not true possessors of freedom, but rather given a \textit{superficial} freedom ultimately bound by rationality. Kant argues that freedom lies in humanity's unique ability to act autonomously, free of external influence, ``...what, then, can freedom of the will be other than autonomy, that is, the will's property of being a law to itself?''. Additionally, Kant defines the \textit{will} as having the exceptional property ``of being a law to itself[.]'' Kant proves directly that in having a will, we are presupposed to have freedom, as the capacity for rational self-deliberation induced by will is the exact definition of free
dom. In being free, we are afforded the responsibility of weighing the moral value of our actions, and in having a will, we choose the action that arises truly from duty rather than inclination or conformity.  However, it is equivalently implied that a will does \textit{not} stem from freedom, but rather comes naturally from the human. So, even if a person is \textit{superficially free} in the sense that they are able to weigh the moral value of their actions, it is not exactly clear that their will is unbound and that they adhere to these actions through their maxims. Should one choose to act in a method not adherent to duty, whether in inclination or conformity, then they fall directly into contradiction, as one cannot be truly free if their will is bound. In a perfectly rational world, one would never have to act in contrary to their will, but we unfortunately don't practice ethics in a perfectly rational world.\\
Returning to the example of the captured victim, the victim cannot, in good conscience, adhere to his maxims without immediate, objectively evil consequence. However, he is still superficially free to weigh the moral value of his actions and may come to the conclusion that, under Kantian ethics, he must abide by his maxims and willfully offer his own life and his family's lives to his kidnappers. Should he choose the more contemporary and widely accepted ethical view to tell a lie and save his own life and those associated with it, under Kant, he suddenly no longer possesses freedom. In this way, one can see that freedom is only achieved through a binding of choice that ultimately results in a tragic end for our victim in this morally difficult situation.\\
Similarly, for the well owner, there is simply no outcome in which he attains any freedom under Kantian ethics, as both choices are contrary to his maxim of lying only to protect the innocent. He is only afforded the luxury of superficial freedom - repeatedly weighing the moral value of his actions under Kantian ethics, while standing idly by - unable to fulfill his duty to either the village or the refugees.\\
In conclusion, under a realist viewpoint, Kantian ethics as proposed in Kant's \textit{Groundwork} are difficult to apply to society. While the text serves as a rigorous framework for moral value and ethical reasoning, Kant places too strong an emphasis on perfect rationality and universally applicable ethical maxims. Under Kant, one would run into a plethora of situations where utilizing a categorical imperative to determine your course of action would be unwise at best and harmful at worst. Because the maxims determined by categorical imperatives are derived exclusively from \textit{a priori} reasoning, the complexity and unpredictability of the real world in both historic and contemporary times means that Kant's ethical theory is too inflexible and abstract, serving as a tool for discussion of human ethics rather than a true guide to acting with moral value.




% \newpage


% %%%%Title
% \begin{center}
% Notes
% \end{center}


% \setlength{\parindent}{0.5in}

% 1. Danhof includes “Delaware, Maryland, all states north of the Potomac and Ohio rivers, Missouri, and states to its north” when referring to the northern states (11).


% 2. For the purposes of this paper,“science” is defined as it was in nineteenthcentury agriculture: conducting experiments and engaging in research.


% 3. Please note that any direct quotes from the nineteenth century texts are writtenin their original form, which may contain grammar mistakes according to twenty-first century grammar rules.

% %%%%Works cited
% \begin{workscited}

% \bibent
% Allen, R.L. \textit{The American Farm Book; or Compend of Ameri can Agriculture; Being a Practical Treatise on Soils, Manures, Draining, Irrigation, Grasses, Grain, Roots, Fruits, Cotton, Tobacco, Sugar Cane, Rice, and Every Staple Product of the United States with the Best Methods of Planting, Cultivating, and Prep aration for Market.} New York: Saxton, 1849. Print.

% \bibent
% Baker, Gladys L., Wayne D. Rasmussen, Vivian Wiser, and Jane M. Porter. \textit{Century of Service: The First 100 Years of the United States Department of Agriculture.}[Federal Government], 1996. Print.

% \bibent
% Danhof, Clarence H. \textit{Change in Agriculture: The Northern United States, 1820-1870.} Cambridge: Harvard UP, 1969. Print.


% \end{workscited}

\end{flushleft}
\end{document}
\}